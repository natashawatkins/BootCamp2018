\documentclass[letterpaper,12pt]{article}
\usepackage{parskip}
\usepackage{amsmath}
\usepackage{amssymb}
\usepackage{mathrsfs}

\setlength{\parindent}{0pt}


\begin{document}

\textbf{\large Problem Set 1}

Natasha Watkins

\vspace{5mm}

\textbf{Exercise 1.3}

$\mathcal{G}_1$ is not an algebra.
\begin{itemize}
  \item Take $B \in \mathcal{G}_1$. Its complement, $B^c$, is closed, and therefore not in $\mathcal{G}_1$. Hence, $\mathcal{G}_1$ is not closed under complements.
\end{itemize}

$\mathcal{G}_2$ is an algebra.

$\mathcal{G}_3$ is a $\sigma$-algebra.


\textbf{Exercise 1.7}

$\mathcal{A}$ is a $\sigma$-algebra of $X$. $\{\emptyset, X\} \subset \mathcal{A} \subset \mathcal{P}(X)$.

Part 1:

\begin{itemize}
  \item By definition, $\emptyset \in \mathcal{A}$.
  \item $\emptyset \in \mathcal{A}$ implies $\emptyset^c = X \in \mathcal{A}$.
  \item So $\{\emptyset, X\}$ is the smallest $\sigma$-algebra of X.
\end{itemize}

Part 2:

\begin{itemize}
  \item Suppose $\mathcal{P} (X)$ is not the largest $\sigma$-algebra on $X$.
  \item Then there is $B \in X$ such that $\mathcal{P} (X) \cup B$ is a $\sigma$-algebra of $X$, where $B \not \in \mathcal{P} (X)$.
  \item But $B \in X$, so $B \in \mathcal{P} (X)$. This is a contradiction, so
  $\mathcal{P} (X)$ must be the largest $\sigma$-algebra.
\end{itemize}

\textbf{Exercise 1.10}

$\{S_\alpha\}$ is a family of $\sigma$-algebras on $X$.

\begin{itemize}
  \item As $\emptyset \in \mathcal{S}_\alpha$ for all $\alpha$,
  $\emptyset \in \cap_\alpha \mathcal{S}_\alpha$.
  \item Given $A \in \cap_\alpha \mathcal{S}_\alpha$, $A \in\mathcal{S}_\alpha$ for all $\alpha$ and $A^c \in \mathcal{S}_\alpha$ for all $\alpha$. So $A^c \in \cap_\alpha \mathcal{S}_\alpha$.
  \item Given $A_1, A_2, \cdots \in \cap_\alpha \mathcal{S}_\alpha$, $A_1, A_2, \cdots \in \mathcal{S}_\alpha$ for all $\alpha$. So $\cup_{n=1}^\infty A_n \in \mathcal{S}_\alpha$ for all $\alpha$. Hence, $\cup_{n=1}^\infty A_n \in \cap_\alpha \mathcal{S}_\alpha$.
\end{itemize}

Therefore, $\cap_\alpha \mathcal{S}_\alpha$ is a $\sigma$-algebra.

\textbf{Exercise 1.17}

\begin{itemize}
  \item $A, B \in \mathcal{S}$, $A \subset B$. $A \cup (B \cap A^c) = B$. As $A$ and $B \cap A^c$ are disjoint, $\mu(A) + \mu(B \cap A^c) = \mu(B)$. This implies $\mu(A) \leq \mu(B)$.
  \item Suppose we form a sequence
        $$
        \begin{align}
        B_1 &= A_1 \\
        B_2 &= A_2 \cap A_1^c \\
        B_3 &= A_3 \cap (A_2^c \cup A_3^c) \\
        \vdots \\
        B_n & = A_n \setminus \bigcup_{i=1}^\infty A_i^
        \end{align}
        $$
        ${B_i}$ is a sequence of disjoint sets, so $\mu(\cup_{i=1}^\infty{B_i}) = \sum_{i=1}^{\infty} \mu(B_i)$.
  \item As $B_i \subset A_i \ \forall i$, by monotonicity, $\sum_{i=1}^{\infty} \mu(B_i) \leq \sum_{i=1}^{\infty} \mu(A_i)$.
  \item Therefore $\mu(\cup_{i=1}^\infty{A_i}) \leq \sum_{i=1}^{\infty} \mu(A_i)$
\end{itemize}

\textbf{Exercise 1.18}

\begin{itemize}
  \item $\lambda(\emptyset) = \mu(\emptyset \cap B) = \mu(\emptyset) = 0$.
  \item $\lambda(\cup_{i=1}^\infty A_i) = \mu((\cup_i^\infty A_i \cap B) =
          \mu(\cup_{i=1}^\infty (A_i \cap B))$
  \item As $ \{ A_i \} $ is collection of disjoint sets,
  $$
  \mu(\cup_{i=1}^\infty (A_i \cap B)) = \sum_{i=1}^\infty \mu(A_i \cap B) = \sum_{i=1}^\infty \lambda(A_i)
  $$
        so $\lambda(A)$ is a measure on $(X, \mathcal S)$.
\end{itemize}

\textbf{Exercise 1.20}

\begin{itemize}
  \item $\cap_{i=1}^n A_i = A_n$
  \item $\mu (\cap_{i=1}^\infty A_i) = 
         \lim_{n \to \infty} \mu (\cap_{i=1}^n A_i)
        = \lim_{n \to \infty} \mu(A_n)$
\end{itemize}

\textbf{Exercise 2.10}

\begin{itemize}
  \item $\mu^*$ is an outer measure, so it satisfies $\mu^*(\cup_{i=1}^\infty A_i) \leq \sum_{i=1}^\infty \mu^*(A_i)$
  \item Considering $B \cap E$ and $B \cup E^c$, by countably subadditivity, $\mu^*((B \cap E) \cup (B \cap E^C)) = \mu^*(B) \leq 
  \mu(B \cap E) + \mu(B \cap E^c)$
  \item Combining this with (*), we see that $\mu^*(B) = 
  \mu(B \cap E) + \mu(B \cap E^c)$
\end{itemize}

\textbf{Exercise 2.14}

\begin{itemize}
  \item An open set $(a, b) \subset \mathbb R$ can be expressed as
        $$
        \bigcup_{n=N}^{\infty} \Big(a, b - \frac{1}{n} \Big]
        $$
        so (a, b) is in $\sigma(\mathcal A)$, as $\sigma(\mathcal A)$ is closed under countable unions
  \item An open interval can be expressed as a disjoint union of open intervals, so any arbitrary open set $O$ is also in $\sigma(\mathcal A)$
  \item So $\sigma(\mathcal O) \subset \sigma(\mathcal A) \subset \mathcal M$, where $\sigma(\mathcal O) = \mathcal B(\mathbb R)$ 
\end{itemize}

\textbf{Exercise 3.1}

\begin{itemize}
  \item Consider a countable set $A \subset \mathbb R$, with $\{a_i\}_{i=1}^n \in A$
  \item Given $\epsilon < 0$, we can construct an open interval around $a_i$ as
        $$
        A_i = (a_i - \frac{\epsilon}{2^i}, a_i + \frac{\epsilon}{2^i})
        $$
  \item So $\mu(A_i) = \frac{\epsilon}{2^i}$ and $\mu(\cup_{i=1}^n A_i) = \sum_{i=1}^n \mu(A_i) = \epsilon$
  \item As $A \subset \cup_{i=1}^n A_i$ and $\epsilon$ is arbitrarily chosen, $\mu(A) = 0$
\end{itemize}

\textbf{Exercise 3.4}

$\{ x \in X: f(x) \leq a \}$

\begin{itemize}
  \item Consider $f(x) \in (-\infty, a)$
  \item As $\mathcal M$ is closed under complements, $(-\infty, a)^c = [a, \infty)$
  \item $(a, \infty)$ can be written as $\bigcup_{n=N}^\infty[a - \frac{1}{n}, \infty)$, which is in $\mathcal M$
  \item $(a, \infty)^c$ is $(-\infty, a]$ which is in $\mathcal M$ given closedness under complements
  \item In the other direction, $(-\infty, a)$ can be expressed as
      $$
      \bigcup_{n=N}^{\infty} \Big(-\infty, a - \frac{1}{n} \Big]
      $$
  \item As $\mathcal M$ is closed under countable unions, $f(x) \in (-\infty, a)$ is equivalent to $f(x) \in (-\infty, a]$
        
\end{itemize}

$\{ x \in X: f(x) > a \}$

\begin{itemize}
  \item Consider $f(x) \in (a, \infty)$
  \item As $\mathcal M$ is closed under complements, $(a, \infty)^c = (-\infty, a]$ is contained in $\mathcal M$ 
  \item So $f(x) \in (a, \infty)$ is equivalent to $f(x) \in (-\infty, a]$. As shown above, this is equivalent to $f(x) \in (-\infty, a)$
\end{itemize}

$\{ x \in X: f(x) \geq a \}$

\begin{itemize}
  \item Consider $f(x) \in [a, \infty)$
  \item As $\mathcal M$ is closed under complements, $[a, \infty)^c = (-\infty, a)$ is contained in $\mathcal M$ 
  \item So $f(x) \in (a, \infty)$ is equivalent to $f(x) \in (-\infty, a)$
\end{itemize}

\textbf{Exercise 4.13}

As $||f|| = f^+ + f^- < M$, then $f^+ < \M$ and $f^- < M$. So both $\int_E f^+ d \mu$ and $\int_E f^- d \mu$ are finite, so $f \in \mathscr{L}^1(\mu, E)$


\end{document}