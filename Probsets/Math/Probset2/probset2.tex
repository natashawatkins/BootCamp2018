\documentclass[letterpaper,12pt]{article}
\usepackage{parskip}
\usepackage{amsmath}
\usepackage{amssymb}
\usepackage{mathrsfs}
\usepackage{gensymb}
\usepackage{enumerate}
\usepackage{fullpage}
\usepackage{setspace}
\onehalfspacing

\setlength{\parindent}{0pt}
\newcommand{\vect}[1]{\mathbf{#1}}
\newcommand*\conj[1]{\bar{#1}}


\begin{document}

\textbf{\large Problem Set 2}

Natasha Watkins

\vspace{5mm}

\textbf{Exercise 3.1}

\underline{Part i)}

We can write
\begin{equation}
  \label{eq1}
\| \vect{x} + \vect{y} \|^2 = \| \vect{x} \|^2 + \langle \vect{x}, \vect{y} \rangle +
\langle \vect{y}, \vect{x} \rangle + \| \vect{y} \|^2
= \| \vect{x} \|^2 + 2 \langle \vect{x}, \vect{y} \rangle + \| \vect{y} \|^2 
\end{equation}

\begin{math}
\text{as } \langle \vect{y}, \vect{x} \rangle 
= \overline{\langle \vect{x}, \vect{y} \rangle}
= \langle \vect{x}, \vect{y} \rangle \ \text{ in } \mathbb R^n.
\end{math}

Using $\langle \vect{x}, \vect{-y} \rangle = (-1)\langle \vect{x}, \vect{y} \rangle$, we can write

\begin{equation}
  \label{eq2}
\| \vect{x} - \vect{y} \|^2 = \| \vect{x} \|^2 - \langle \vect{x}, \vect{y} \rangle -
\langle \vect{y}, \vect{x} \rangle + \| \vect{y} \|^2
= \| \vect{x} \|^2 - 2 \langle \vect{x}, \vect{y} \rangle + \| \vect{y} \|^2 
\end{equation}

Combining,
\begin{equation*}
  \frac{1}{4} (\| \vect{x} + \vect{y} \|^2 - \| \vect{x} - \vect{y} \|^2)
  = \frac{1}{4}(4 \langle \vect{x}, \vect{y} \rangle) = \langle \vect{x}, \vect{y} \rangle
\end{equation*}

\underline{Part ii)}

Combining equations \ref{eq1} and \ref{eq2}, we find

\begin{equation*}
  \frac{1}{2} (\| \vect{x} + \vect{y} \|^2 + \| \vect{x} - \vect{y} \|^2)
  = \frac{1}{2}(2 \langle \vect{x}, \vect{y} \rangle) = \langle \vect{x}, \vect{y} \rangle
\end{equation*}

\textbf{Exercise 3.3}

\underline{Part i)}

\begin{math}
  \langle \vect{x}, \vect{x}^5 \rangle = \int_0^1 x^6 dx = \frac{x^7}{7} \Big|_0^1 = \frac{1}{7} \\
  \langle \vect{x}, \vect{x} \rangle = \int_0^1 x^2 dx = \frac{x^3}{3} \Big|_0^1 = \frac{1}{3} 
  \implies \|\vect{x}\| = \sqrt{\frac{1}{3}} \\
  \langle \vect{x}^5, \vect{x}^5 \rangle = \int_0^1 x^{10} dx = \frac{x^{11}}{11} \Big|_0^{1} = \frac{1}{11} \implies \|\vect{x}^5\| = \sqrt{\frac{1}{11}} \\
  
  \cos(\theta) = \frac{\sqrt{3} \sqrt{11}}{7} = \frac{\sqrt{33}}{7}
  \implies \theta \approx 35 \degree
  
\end{math}
  
\underline{Part ii)}

\begin{math}
  \langle \vect{x}^2, \vect{x}^4 \rangle = \int_0^1 x^6 dx = \frac{x^7}{7} \Big|_0^1 = \frac{1}{7} \\
  \langle \vect{x}^2, \vect{x}^2 \rangle = \int_0^1 x^4 dx = \frac{x^5}{5} \Big|_0^1 = \frac{1}{5} 
  \implies \|\vect{x}^2\| = \sqrt{\frac{1}{5}} \\
  \langle \vect{x}^4, \vect{x}^4 \rangle = \int_0^1 x^{8} dx = \frac{x^{9}}{9} \Big|_0^{1} = \frac{1}{9} \implies \|\vect{x}^4\| = \sqrt{\frac{1}{9}} \\
  
  \cos(\theta) = \frac{\sqrt{9} \sqrt{5}}{7} = \frac{\sqrt{45}}{7}
  \implies \theta \approx 17 \degree
  
\end{math}

\textbf{Exercise 3.8}

i)

\begin{math}
\frac{1}{\pi} \int_{-\pi}^{\pi} \cos(t) \sin(t) dt &= 0    \\
\frac{1}{\pi} \int_{-\pi}^{\pi} \cos(t) \cos(2t) dt &= 0   \\
\frac{1}{\pi} \int_{-\pi}^{\pi} \cos(t) \sin(2t) dt &= 0   \\
\frac{1}{\pi} \int_{-\pi}^{\pi} \cos(2t) \sin(t) dt &= 0   \\
\frac{1}{\pi} \int_{-\pi}^{\pi} \cos(t) \cos(t) dt &= 1    \\
\frac{1}{\pi} \int_{-\pi}^{\pi} \sin(t) \sin(t) dt &= 1    \\
\frac{1}{\pi} \int_{-\pi}^{\pi} \sin(2t) \sin(2t) dt &= 1  \\
\end{math}
Therefore, $S$ is an orthonormal set.

        
ii)
\begin{math}  
\|t\| = \sqrt{\langle t, t \rangle} = \int_{-\pi}^{\pi} t^2 dt
= \frac{t^3}{3} \Big|^\pi_{-\pi} = \frac{2}{3}\pi^2
\end{math}

iii)
\begin{math}
  \text{proj}_X(\cos(3t)) = \sum_{i=1}^m \langle \vect{x}_i, \cos(3t) \rangle 
  \vect{x_i} = 0
\end{math}

iv)
\begin{math}
  \text{proj}_X(t) = \sum_{i=1}^m \langle \vect{x}_i, t \rangle 
  \vect{x_i} = 1
\end{math}

\textbf{Exercise 3.9}

By Theorem 3.2.15, a matrix $Q$ is orthonormal if and only if $Q^HQ = QQ^H = 1$.

The rotation matrix is given by
\begin{align*}
  R_\theta = 
  \begin{bmatrix}
    \cos \theta & -\sin \theta \\
    \sin \theta & \cos \theta \\
  \end{bmatrix}
\end{align*}

Calculating $R_{\theta}R$, we find
\begin{align*}
  R_{\theta}R = 
  \begin{bmatrix}
    \cos \theta & -\sin \theta \\
    \sin \theta & \cos \theta \\
  \end{bmatrix}
  &=
  \begin{bmatrix}
    \cos \theta & \sin \theta \\
    -\sin \theta & \cos \theta \\
  \end{bmatrix}
  \begin{bmatrix}
    \cos \theta & -\sin \theta \\
    \sin \theta & \cos \theta \\
  \end{bmatrix}
  \\ &=
  \begin{bmatrix}
    \cos^2 \theta + \sin^2 \theta & \cos \theta \sin \theta - \sin \theta \cos \theta \\
    \sin \theta \cos \theta - \cos \theta \sin \theta & \sin^2 \theta + \cos^2 \theta \\
  \end{bmatrix}
  \\ &=
  \begin{bmatrix}
    1 & 0 \\
    0 & 1 \\
  \end{bmatrix}
  = \ $I$
\end{align*}
 
So $R_\theta$ is an orthonormal transformation.

\textbf{Exercise 3.10}

\underline{Part i)}

Assume $Q$ is orthonormal, which implies $\langle \vect{e}_i, \vect{e}_j \rangle = \langle Q \vect{e}_i,  Q \vect{e}_j \rangle$.
\begin{align*}
  \vect{e}_i^H \vect{e}_j &= 
  \langle \vect{e}_i, \vect{e}_j \rangle \\
  &= \langle Q \vect{e}_i,  Q \vect{e}_j \rangle \\
  &= (Q \vect{e}_i)^H  (Q \vect{e}_j) \\
  &= \vect{e}_i^H Q^H  Q \vect{e}_j
\end{align*}

$\vect{e}_i^H \vect{e}_j = \vect{e}_i^H Q^H  Q \vect{e}_j$ only if $Q^HQ = I$.

\end{document}