% !TeX spellcheck = en_GB
\documentclass[letterpaper,12pt]{article}
\usepackage{textcomp}
\usepackage{parskip}
\usepackage{amsmath}
\usepackage{amssymb}
\usepackage{mathrsfs}
\usepackage{gensymb}
\usepackage{enumerate}
\usepackage{fullpage}
\usepackage{setspace}
\usepackage{array}
\usepackage{threeparttable}
\usepackage{graphicx}
\usepackage{float}

\onehalfspacing

\setlength{\parindent}{0pt}
\newcommand{\vect}[1]{\mathbf{#1}}
\newcommand*\conj[1]{\bar{#1}}
\setlength{\belowcaptionskip}{-80pt}


\begin{document}

\textbf{\large Problem Set 6}

Natasha Watkins

\vspace{5mm}

\textbf{Exercise 9.1}

An unconstrained linear function can be written as $f(\vect{x}) = \vect{b}^T \vect{x} + c$. In the case where $\vect{b} = \vect{0}$, the function is constant $Df(\vect{x}) = 0 \ \forall \ \vect{x}$, and therefore the minimum is equal to $c$. In the case where $\vect{b} \neq \vect{0}$, $Df(\vect{x}) = \vect{b}^T \ \forall \ \vect{x}$, and therefore there is no minimum.

\textbf{Exercise 9.2}
\begin{align*}
\| A \vect{x} - \vect{b} \|_2
&= (A \vect{x} - \vect{b})^T (A \vect{x} - \vect{b}) \\
&= \vect{x}^T A^T A \vect{x} - \vect{x}^T A^T \vect{b} - \vect{b}^T A \vect{x} + \vect{b}^T \vect{b} \\
&= \vect{x}^T A^T A \vect{x} - 2\vect{b}^T A \vect{x}  + \vect{b}^T \vect{b}
\quad \quad \text{as } \vect{b}^T A \vect{x} = \vect{x}^T A^T \vect{b} \text{ is scalar}
\end{align*}
As $\vect{b}^T \vect{b}$ is a constant, this term can be dropped. As $A$ is symmetric and positive definite, we can write
\begin{align*}
2A^T A \vect{x} - 2\vect{b}^T A \vect{x}
\end{align*}
Setting this equal to 0 and transposing, we find $A^T A = A^T \vect{b}$.


\textbf{Exercises 9.5 - 9.9}

See Jupyter notebook.

\textbf{Exercise 9.10}
\begin{align*}
f'(\vect{x}) &= \vect{x}^T Q - \vect{b}^T \\
f''(\vect{x}) &= Q^T
\end{align*}
Setting $f'(\vect{x}) = \vect{0}$, we find $\vect{x}^* = Q^{-1} \vect{b}$.

Using Newton's method, given $\vect{x}_0$
\begin{align*}
\vect{x}_1 &= \vect{x}_0 - (Q^T)^{-1} (\vect{x}_0^T Q - \vect{b}^T)^T \\
&= \vect{x}_0 - Q^{-1} Q  \vect{x}_0 - Q^{-1}\vect{b} \quad \quad \text{as } Q^T = Q \\
&= Q^{-1} \vect{b}
\end{align*}

\textbf{Exercise 9.12}

Given an eigenvector $\vect{x}_i$ of A, where $A \vect{x}_i = \lambda_i \vect{x}$, we have
\begin{align*}
B \vect{x}_i &= (A + \mu I) \vect{x}_i \\
&= A\vect{x}_i + \mu \vect{x}_i \\
&= \lambda_i \vect{x}_i + \mu \vect{x}_i \\
&= (\lambda_i + \mu) \vect{x}_i
\end{align*}
Therefore $A$ and $B$ have the same eigenvectors, and $B$ has eigenvalues of the form $\lambda_i + \mu$.

\end{document}