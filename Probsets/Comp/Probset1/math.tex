\documentclass[letterpaper,12pt]{article}
\usepackage{parskip}
\usepackage{amsmath}
\usepackage{amssymb}

\setlength{\parindent}{0pt}


\begin{document}

\textbf{\large Problem Set 1}

Natasha Watkins

\vspace{5mm}

\textbf{Exercise 1.3}

$\mathcal{G}_1$ is not an algebra.
\begin{itemize}
  \item Take $B \in \mathcal{G}_1$. Its complement, $B^c$, is closed, and therefore not in $\mathcal{G}_1$. Hence, $\mathcal{G}_1$ is not closed under complements.
\end{itemize}

$\mathcal{G}_2$ is an algebra.

$\mathcal{G}_3$ is a $\sigma$-algebra.


\textbf{Exercise 1.7}

$\mathcal{A}$ is a $\sigma$-algebra of $X$. $\{\emptyset, X\} \subset \mathcal{A} \subset \mathcal{P}(X)$.

Part 1:

\begin{itemize}
  \item By definition, $\emptyset \in \mathcal{A}$.
  \item $\emptyset \in \mathcal{A}$ implies $\emptyset^c = X \in \mathcal{A}$.
  \item So $\{\emptyset, X\}$ is the smallest $\sigma$-algebra of X.
\end{itemize}

Part 2:

\begin{itemize}
  \item Suppose $\mathcal{P} (X)$ is not the largest $\sigma$-algebra on $X$.
  \item Then there is $B \in X$ such that $\mathcal{P} (X) \cup B$ is a $\sigma$-algebra of $X$, where $B \not \in \mathcal{P} (X)$.
  \item But $B \in X$, so $B \in \mathcal{P} (X)$. This is a contradiction, so
  $\mathcal{P} (X)$ must be the largest $\sigma$-algebra.
\end{itemize}

\textbf{Exercise 1.10}

$\{S_\alpha\}$ is a family of $\sigma$-algebras on $X$.

\begin{itemize}
  \item As $\emptyset \in \mathcal{S}_\alpha$ for all $\alpha$,
  $\emptyset \in \cap_\alpha \mathcal{S}_\alpha$.
  \item Given $A \in \cap_\alpha \mathcal{S}_\alpha$, $A \in\mathcal{S}_\alpha$ for all $\alpha$ and $A^c \in \mathcal{S}_\alpha$ for all $\alpha$. So $A^c \in \cap_\alpha \mathcal{S}_\alpha$.
  \item Given $A_1, A_2, \cdots \in \cap_\alpha \mathcal{S}_\alpha$, $A_1, A_2, \cdots \in \mathcal{S}_\alpha$ for all $\alpha$. So $\cup_{n=1}^\infty A_n \in \mathcal{S}_\alpha$ for all $\alpha$. Hence, $\cup_{n=1}^\infty A_n \in \cap_\alpha \mathcal{S}_\alpha$.
\end{itemize}

Therefore, $\cap_\alpha \mathcal{S}_\alpha$ is a $\sigma$-algebra.

\textbf{Exercise 1.17}

\begin{itemize}
  \item $A, B \in \mathcal{S}$, $A \subset B$. $A \cup (B \cap A^c) = B$. As $A$ and $B \cap A^c$ are disjoint, $\mu(A) + \mu(B \cap A^c) = \mu(B)$. This implies $\mu(A) \leq \mu(B)$.
\end{itemize}

\textbf{Exercise 1.18}

\begin{itemize}
  \item $\lambda(\emptyset) = \mu(\emptyset \cap B) = \mu(\emptyset) = 0$.
\end{itemize}

\textbf{Exercise 1.20}

\begin{itemize}
  \item $\cap_{i=1}^n A_i = A_n$
  \item $\mu (\cap_{i=1}^\infty A_i) = 
         \lim_{n \to \infty} \mu (\cap_{i=1}^n A_i)
        = \lim_{n \to \infty} \mu(A_n)$
\end{itemize}

\end{document}