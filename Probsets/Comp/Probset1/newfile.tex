\documentclass[letterpaper,12pt]{article}
\usepackage{parskip}
\usepackage{amsmath}
\usepackage{amssymb}


\begin{document}

\textbf{\large problem set 1}

natasha watkins

\vspace{5mm}

\noindent\textbf{exercise 1.7}

$\mathcal{a}$ is a $\sigma$-algebra of $x$. $\{\emptyset, x\} \subset \mathcal{a} \subset \mathcal{p}(x)$.

part 1:

\begin{itemize}
  \item take $a \in \mathcal{a}$. then $a^c \in \mathcal{a}$. so $a \cup a^c = x \in \mathcal{a}$.
  \item $x \in \mathcal{a}$ implies $x^c = \emptyset \in \mathcal{a}$.
  \item so $\{\emptyset, x\}$ is the smallest $\sigma$-algebra of x.
\end{itemize}

part 2:

\noindent\textbf{exercise 1.10}

$\{s_\alpha\}$ is a family of $\sigma$-algebras on $x$.

\begin{itemize}
  \item as $\emptyset \in \mathcal{s}_\alpha$ for all $\alpha$,
  $\emptyset \in \cap_\alpha \mathcal{s}_\alpha$.
  \item given $a \in \cap_\alpha \mathcal{s}_\alpha$, $a \in\mathcal{s}_\alpha$ for all $\alpha$ and $a^c \in \mathcal{s}_\alpha$ for all $\alpha$. so $a^c \in \cap_\alpha \mathcal{s}_\alpha$.
  \item given $a_1, a_2, \cdots \in \cap_\alpha \mathcal{s}_\alpha$, $a_1, a_2, \cdots \in \mathcal{s}_\alpha$ for all $\alpha$. so $\cup_{n=1}^\infty a_n \in \mathcal{s}_\alpha$ for all $\alpha$. hence, $\cup_{n=1}^\infty a_n \in \cap_\alpha \mathcal{s}_\alpha$.
\end{itemize}

therefore, $\cap_\alpha \mathcal{s}_\alpha$ is a $\sigma$-algebra.

\noindent\textbf{exercise 1.18}

\begin{itemize}
  \item $\lambda(\emptyset) = \mu(\emptyset \cap b) = \mu(\emptyset) = 0$.
\end{itemize}

\end{document}